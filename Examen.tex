% Options for packages loaded elsewhere
\PassOptionsToPackage{unicode}{hyperref}
\PassOptionsToPackage{hyphens}{url}
%
\documentclass[
]{article}
\usepackage{amsmath,amssymb}
\usepackage{iftex}
\ifPDFTeX
  \usepackage[T1]{fontenc}
  \usepackage[utf8]{inputenc}
  \usepackage{textcomp} % provide euro and other symbols
\else % if luatex or xetex
  \usepackage{unicode-math} % this also loads fontspec
  \defaultfontfeatures{Scale=MatchLowercase}
  \defaultfontfeatures[\rmfamily]{Ligatures=TeX,Scale=1}
\fi
\usepackage{lmodern}
\ifPDFTeX\else
  % xetex/luatex font selection
\fi
% Use upquote if available, for straight quotes in verbatim environments
\IfFileExists{upquote.sty}{\usepackage{upquote}}{}
\IfFileExists{microtype.sty}{% use microtype if available
  \usepackage[]{microtype}
  \UseMicrotypeSet[protrusion]{basicmath} % disable protrusion for tt fonts
}{}
\makeatletter
\@ifundefined{KOMAClassName}{% if non-KOMA class
  \IfFileExists{parskip.sty}{%
    \usepackage{parskip}
  }{% else
    \setlength{\parindent}{0pt}
    \setlength{\parskip}{6pt plus 2pt minus 1pt}}
}{% if KOMA class
  \KOMAoptions{parskip=half}}
\makeatother
\usepackage{xcolor}
\usepackage[margin=1in]{geometry}
\usepackage{color}
\usepackage{fancyvrb}
\newcommand{\VerbBar}{|}
\newcommand{\VERB}{\Verb[commandchars=\\\{\}]}
\DefineVerbatimEnvironment{Highlighting}{Verbatim}{commandchars=\\\{\}}
% Add ',fontsize=\small' for more characters per line
\usepackage{framed}
\definecolor{shadecolor}{RGB}{248,248,248}
\newenvironment{Shaded}{\begin{snugshade}}{\end{snugshade}}
\newcommand{\AlertTok}[1]{\textcolor[rgb]{0.94,0.16,0.16}{#1}}
\newcommand{\AnnotationTok}[1]{\textcolor[rgb]{0.56,0.35,0.01}{\textbf{\textit{#1}}}}
\newcommand{\AttributeTok}[1]{\textcolor[rgb]{0.13,0.29,0.53}{#1}}
\newcommand{\BaseNTok}[1]{\textcolor[rgb]{0.00,0.00,0.81}{#1}}
\newcommand{\BuiltInTok}[1]{#1}
\newcommand{\CharTok}[1]{\textcolor[rgb]{0.31,0.60,0.02}{#1}}
\newcommand{\CommentTok}[1]{\textcolor[rgb]{0.56,0.35,0.01}{\textit{#1}}}
\newcommand{\CommentVarTok}[1]{\textcolor[rgb]{0.56,0.35,0.01}{\textbf{\textit{#1}}}}
\newcommand{\ConstantTok}[1]{\textcolor[rgb]{0.56,0.35,0.01}{#1}}
\newcommand{\ControlFlowTok}[1]{\textcolor[rgb]{0.13,0.29,0.53}{\textbf{#1}}}
\newcommand{\DataTypeTok}[1]{\textcolor[rgb]{0.13,0.29,0.53}{#1}}
\newcommand{\DecValTok}[1]{\textcolor[rgb]{0.00,0.00,0.81}{#1}}
\newcommand{\DocumentationTok}[1]{\textcolor[rgb]{0.56,0.35,0.01}{\textbf{\textit{#1}}}}
\newcommand{\ErrorTok}[1]{\textcolor[rgb]{0.64,0.00,0.00}{\textbf{#1}}}
\newcommand{\ExtensionTok}[1]{#1}
\newcommand{\FloatTok}[1]{\textcolor[rgb]{0.00,0.00,0.81}{#1}}
\newcommand{\FunctionTok}[1]{\textcolor[rgb]{0.13,0.29,0.53}{\textbf{#1}}}
\newcommand{\ImportTok}[1]{#1}
\newcommand{\InformationTok}[1]{\textcolor[rgb]{0.56,0.35,0.01}{\textbf{\textit{#1}}}}
\newcommand{\KeywordTok}[1]{\textcolor[rgb]{0.13,0.29,0.53}{\textbf{#1}}}
\newcommand{\NormalTok}[1]{#1}
\newcommand{\OperatorTok}[1]{\textcolor[rgb]{0.81,0.36,0.00}{\textbf{#1}}}
\newcommand{\OtherTok}[1]{\textcolor[rgb]{0.56,0.35,0.01}{#1}}
\newcommand{\PreprocessorTok}[1]{\textcolor[rgb]{0.56,0.35,0.01}{\textit{#1}}}
\newcommand{\RegionMarkerTok}[1]{#1}
\newcommand{\SpecialCharTok}[1]{\textcolor[rgb]{0.81,0.36,0.00}{\textbf{#1}}}
\newcommand{\SpecialStringTok}[1]{\textcolor[rgb]{0.31,0.60,0.02}{#1}}
\newcommand{\StringTok}[1]{\textcolor[rgb]{0.31,0.60,0.02}{#1}}
\newcommand{\VariableTok}[1]{\textcolor[rgb]{0.00,0.00,0.00}{#1}}
\newcommand{\VerbatimStringTok}[1]{\textcolor[rgb]{0.31,0.60,0.02}{#1}}
\newcommand{\WarningTok}[1]{\textcolor[rgb]{0.56,0.35,0.01}{\textbf{\textit{#1}}}}
\usepackage{graphicx}
\makeatletter
\def\maxwidth{\ifdim\Gin@nat@width>\linewidth\linewidth\else\Gin@nat@width\fi}
\def\maxheight{\ifdim\Gin@nat@height>\textheight\textheight\else\Gin@nat@height\fi}
\makeatother
% Scale images if necessary, so that they will not overflow the page
% margins by default, and it is still possible to overwrite the defaults
% using explicit options in \includegraphics[width, height, ...]{}
\setkeys{Gin}{width=\maxwidth,height=\maxheight,keepaspectratio}
% Set default figure placement to htbp
\makeatletter
\def\fps@figure{htbp}
\makeatother
\setlength{\emergencystretch}{3em} % prevent overfull lines
\providecommand{\tightlist}{%
  \setlength{\itemsep}{0pt}\setlength{\parskip}{0pt}}
\setcounter{secnumdepth}{-\maxdimen} % remove section numbering
\ifLuaTeX
  \usepackage{selnolig}  % disable illegal ligatures
\fi
\usepackage{bookmark}
\IfFileExists{xurl.sty}{\usepackage{xurl}}{} % add URL line breaks if available
\urlstyle{same}
\hypersetup{
  pdftitle={Examen},
  pdfauthor={Jorge Alfredo Suazo Victoria},
  hidelinks,
  pdfcreator={LaTeX via pandoc}}

\title{Examen}
\author{Jorge Alfredo Suazo Victoria}
\date{2024-09-05}

\begin{document}
\maketitle

\begin{enumerate}
\def\labelenumi{\arabic{enumi}.}
\tightlist
\item
  Si dos portadores del gen del albinismo se casan, existe una
  probabilidad de 0.1 de que los hijos hereden el. Si la pareja tiene 3
  hijos, calcular:
\end{enumerate}

\begin{enumerate}
\def\labelenumi{\alph{enumi})}
\tightlist
\item
  La probabilidad de que al menos uno tenga el gen;
\end{enumerate}

\begin{Shaded}
\begin{Highlighting}[]
\NormalTok{p }\OtherTok{=}\NormalTok{ .}\DecValTok{1}
\NormalTok{hijos }\OtherTok{=} \DecValTok{3}

\NormalTok{(}\AttributeTok{atl1 =} \DecValTok{1} \SpecialCharTok{{-}} \FunctionTok{dbinom}\NormalTok{(}\DecValTok{0}\NormalTok{,hijos,p))}
\end{Highlighting}
\end{Shaded}

\begin{verbatim}
## [1] 0.271
\end{verbatim}

\begin{enumerate}
\def\labelenumi{\alph{enumi})}
\setcounter{enumi}{1}
\tightlist
\item
  La probabilidad de que exactamente dos tengan el gen;
\end{enumerate}

\begin{Shaded}
\begin{Highlighting}[]
\NormalTok{(}\AttributeTok{ex2 =} \FunctionTok{dbinom}\NormalTok{(}\DecValTok{2}\NormalTok{,}\DecValTok{3}\NormalTok{,p))}
\end{Highlighting}
\end{Shaded}

\begin{verbatim}
## [1] 0.027
\end{verbatim}

\begin{enumerate}
\def\labelenumi{\alph{enumi})}
\setcounter{enumi}{2}
\tightlist
\item
  Numero esperado de hijos con el gen del albinismo
\end{enumerate}

\begin{Shaded}
\begin{Highlighting}[]
\NormalTok{(}\AttributeTok{E =}\NormalTok{ p }\SpecialCharTok{*}\NormalTok{ hijos)}
\end{Highlighting}
\end{Shaded}

\begin{verbatim}
## [1] 0.3
\end{verbatim}

\begin{enumerate}
\def\labelenumi{\arabic{enumi}.}
\setcounter{enumi}{1}
\tightlist
\item
  En un cierto colegio, con una poblacion de 640 estudiantes, el 55\%
  está formado por mujeres, suponga que se toma una muestra de dos
  estudiantes.
\end{enumerate}

\begin{enumerate}
\def\labelenumi{\alph{enumi})}
\tightlist
\item
  ¿Cual es la probabilidad de que se elijan dos mujeres?
\end{enumerate}

\begin{Shaded}
\begin{Highlighting}[]
\NormalTok{total }\OtherTok{=} \DecValTok{640}
\NormalTok{mujeres }\OtherTok{=}\NormalTok{ total }\SpecialCharTok{*}\NormalTok{ .}\DecValTok{55}
\NormalTok{PW }\OtherTok{=}\NormalTok{ .}\DecValTok{55}
\NormalTok{SWdPW }\OtherTok{=}\NormalTok{ (mujeres}\DecValTok{{-}1}\NormalTok{)}\SpecialCharTok{/}\NormalTok{(total}\DecValTok{{-}1}\NormalTok{) }\CommentTok{\# Probabilidad de Que la Segunda sea mujer dado que la primera fue mujer}
\NormalTok{SWdPH }\OtherTok{=}\NormalTok{ (mujeres)}\SpecialCharTok{/}\NormalTok{(total}\DecValTok{{-}1}\NormalTok{) }\CommentTok{\# Probabilidad de que la segunda sea mujer dado que el primero fue hombre}

\NormalTok{(}\AttributeTok{TW =}\NormalTok{ PW }\SpecialCharTok{*}\NormalTok{ SWdPW)}
\end{Highlighting}
\end{Shaded}

\begin{verbatim}
## [1] 0.3021127
\end{verbatim}

\begin{enumerate}
\def\labelenumi{\alph{enumi})}
\setcounter{enumi}{1}
\tightlist
\item
  ¿Cual es la probabilidad de que, al menos una sea mujer?
\end{enumerate}

\begin{Shaded}
\begin{Highlighting}[]
\NormalTok{PH }\OtherTok{\textless{}{-}} \DecValTok{1} \SpecialCharTok{{-}}\NormalTok{ PW  }
\NormalTok{SHdPH }\OtherTok{\textless{}{-}}\NormalTok{ ((total }\SpecialCharTok{*}\NormalTok{ PH) }\SpecialCharTok{{-}} \DecValTok{1}\NormalTok{) }\SpecialCharTok{/}\NormalTok{ (total }\SpecialCharTok{{-}} \DecValTok{1}\NormalTok{)}
\NormalTok{TH }\OtherTok{\textless{}{-}}\NormalTok{ PH }\SpecialCharTok{*}\NormalTok{ SHdPH}
\NormalTok{(Patl1W }\OtherTok{\textless{}{-}} \DecValTok{1} \SpecialCharTok{{-}}\NormalTok{ TH)}
\end{Highlighting}
\end{Shaded}

\begin{verbatim}
## [1] 0.7978873
\end{verbatim}

\begin{enumerate}
\def\labelenumi{\alph{enumi})}
\setcounter{enumi}{2}
\tightlist
\item
  ¿Cual es la probabilidad la primera persona elegida haya sido mujer.
  dado que la segunda fue mujer?
\end{enumerate}

\begin{Shaded}
\begin{Highlighting}[]
\NormalTok{SM }\OtherTok{=}\NormalTok{ SWdPW }\SpecialCharTok{*}\NormalTok{ SWdPH }
\NormalTok{(}\AttributeTok{PWdSW =}\NormalTok{ SWdPW }\SpecialCharTok{*}\NormalTok{ PW }\SpecialCharTok{/}\NormalTok{ SM)}
\end{Highlighting}
\end{Shaded}

\begin{verbatim}
## [1] 0.9984375
\end{verbatim}

\begin{enumerate}
\def\labelenumi{\arabic{enumi}.}
\setcounter{enumi}{2}
\tightlist
\item
  Suponga que una enfermedad se hereda mediante un mecanismo ligado al
  sexo de la persona, de forma que, si la descendencia es masculina,
  entonces la posibilidad de que herede la enfermedad es 50\%, mientras
  que, si la desccendencia es femenina entonces no es posible que se
  herede la enfermedad. Si en esta poblacion el 51,3\% de los
  nacimientos son masculinos, ¿Cual es la probabilidad de que un
  individuo seleccionado al azar sea afectado por la enfermedad?
\end{enumerate}

\begin{Shaded}
\begin{Highlighting}[]
\NormalTok{(}\AttributeTok{p =}\NormalTok{ .}\DecValTok{5} \SpecialCharTok{*}\NormalTok{ .}\DecValTok{513}\NormalTok{)}
\end{Highlighting}
\end{Shaded}

\begin{verbatim}
## [1] 0.2565
\end{verbatim}

\begin{enumerate}
\def\labelenumi{\arabic{enumi}.}
\setcounter{enumi}{3}
\tightlist
\item
  Considere un juego en el que se obtienen ganancias de acuerdo con el
  numero de aguilas que aparecen al lanzar una moneda 3 veces. Sea X la
  variable aleatoria que indica el numero de aguilas en tres
  lanzamientos y sea Y la variable que indica la ganancia del juego en
  pesos, la cual depende del lanzamiento que ocurre en la primera
  aguila. La siguiente tabla muestra la funcion de probabilidad conjunta
  de X y Y.
\end{enumerate}

\begin{Shaded}
\begin{Highlighting}[]
\NormalTok{table }\OtherTok{\textless{}{-}} \FunctionTok{matrix}\NormalTok{(}\FunctionTok{c}\NormalTok{(}\DecValTok{1}\SpecialCharTok{/}\DecValTok{8}\NormalTok{, }\DecValTok{0}\NormalTok{ , }\DecValTok{0}\NormalTok{ ,}\DecValTok{0}\NormalTok{, }\DecValTok{0}\NormalTok{, }\DecValTok{1}\SpecialCharTok{/}\DecValTok{8}\NormalTok{, }\DecValTok{2}\SpecialCharTok{/}\DecValTok{8}\NormalTok{, }\DecValTok{1}\SpecialCharTok{/}\DecValTok{8}\NormalTok{, }\DecValTok{0}\NormalTok{, }\DecValTok{1}\SpecialCharTok{/}\DecValTok{8}\NormalTok{, }\DecValTok{1}\SpecialCharTok{/}\DecValTok{8}\NormalTok{, }\DecValTok{0}\NormalTok{, }\DecValTok{0}\NormalTok{, }\DecValTok{1}\SpecialCharTok{/}\DecValTok{8}\NormalTok{, }\DecValTok{0}\NormalTok{, }\DecValTok{0}\NormalTok{), }\AttributeTok{byrow =}\NormalTok{ T, }\AttributeTok{nrow =} \DecValTok{4}\NormalTok{)}
\FunctionTok{print}\NormalTok{(table)}
\end{Highlighting}
\end{Shaded}

\begin{verbatim}
##       [,1]  [,2]  [,3]  [,4]
## [1,] 0.125 0.000 0.000 0.000
## [2,] 0.000 0.125 0.250 0.125
## [3,] 0.000 0.125 0.125 0.000
## [4,] 0.000 0.125 0.000 0.000
\end{verbatim}

\begin{enumerate}
\def\labelenumi{\alph{enumi})}
\tightlist
\item
  Encuentre la probabilidad de que el numero de aguilas sea igual a 2
\end{enumerate}

\begin{Shaded}
\begin{Highlighting}[]
\NormalTok{(}\FunctionTok{sum}\NormalTok{(table[,}\DecValTok{3}\NormalTok{]))}
\end{Highlighting}
\end{Shaded}

\begin{verbatim}
## [1] 0.375
\end{verbatim}

\begin{enumerate}
\def\labelenumi{\alph{enumi})}
\setcounter{enumi}{1}
\tightlist
\item
  Encuentre la probabilidad de que la ganancia sea positiva.
\end{enumerate}

\begin{Shaded}
\begin{Highlighting}[]
\NormalTok{(}\FunctionTok{sum}\NormalTok{(table[}\FunctionTok{c}\NormalTok{(}\DecValTok{2}\SpecialCharTok{:}\DecValTok{4}\NormalTok{),]))}
\end{Highlighting}
\end{Shaded}

\begin{verbatim}
## [1] 0.875
\end{verbatim}

\begin{enumerate}
\def\labelenumi{\alph{enumi})}
\setcounter{enumi}{2}
\tightlist
\item
  Encuentre la probabilidad de ganar dos pesos dado que se obtuvieron
  dos aguilas
\end{enumerate}

\begin{Shaded}
\begin{Highlighting}[]
\NormalTok{table[}\DecValTok{3}\NormalTok{,}\DecValTok{3}\NormalTok{] }\SpecialCharTok{/} \FunctionTok{sum}\NormalTok{(table[,}\DecValTok{3}\NormalTok{])}
\end{Highlighting}
\end{Shaded}

\begin{verbatim}
## [1] 0.3333333
\end{verbatim}

\begin{enumerate}
\def\labelenumi{\alph{enumi})}
\setcounter{enumi}{3}
\tightlist
\item
  Encuentre la probabilidad de que se hayan obtenido dos aguilas dado
  que la ganancia fue de 1 peso
\end{enumerate}

\begin{Shaded}
\begin{Highlighting}[]
\NormalTok{table[}\DecValTok{2}\NormalTok{,}\DecValTok{3}\NormalTok{]}\SpecialCharTok{/}\FunctionTok{sum}\NormalTok{(table[}\DecValTok{2}\NormalTok{,])}
\end{Highlighting}
\end{Shaded}

\begin{verbatim}
## [1] 0.5
\end{verbatim}

\begin{enumerate}
\def\labelenumi{\arabic{enumi}.}
\setcounter{enumi}{4}
\tightlist
\item
  Sea X la variable aleatoria que indica el nivel de glucosa (mg/dl) en
  personas sanas. Suponga que X \textasciitilde{}
  \(Gamma(\alpha = 90, ~\beta = 85)\) donde \(\alpha\) es el parametro
  de forma y \(\beta\) es el parametro de escala
\end{enumerate}

\begin{enumerate}
\def\labelenumi{\alph{enumi})}
\tightlist
\item
  ¿Qué tan probable es que una persona sana tenga un nivel de glucosa
  por arriba de 100mg/dl?
\end{enumerate}

\begin{Shaded}
\begin{Highlighting}[]
\NormalTok{alpha }\OtherTok{\textless{}{-}} \DecValTok{90}
\NormalTok{beta }\OtherTok{\textless{}{-}}\NormalTok{ .}\DecValTok{85}
\NormalTok{(more }\OtherTok{\textless{}{-}}\NormalTok{ (}\DecValTok{1} \SpecialCharTok{{-}} \FunctionTok{pgamma}\NormalTok{(}\DecValTok{100}\NormalTok{, }\AttributeTok{shape =}\NormalTok{ alpha, }\AttributeTok{scale =}\NormalTok{ beta)))}
\end{Highlighting}
\end{Shaded}

\begin{verbatim}
## [1] 0.003526158
\end{verbatim}

\begin{enumerate}
\def\labelenumi{\alph{enumi})}
\setcounter{enumi}{1}
\tightlist
\item
  ¿Que tan probable es que el nivel se encuentre por debajo de 70?
\end{enumerate}

\begin{Shaded}
\begin{Highlighting}[]
\NormalTok{(less70 }\OtherTok{\textless{}{-}} \FunctionTok{pgamma}\NormalTok{(}\DecValTok{70}\NormalTok{,}\AttributeTok{shape=}\NormalTok{alpha,}\AttributeTok{scale=}\NormalTok{beta))}
\end{Highlighting}
\end{Shaded}

\begin{verbatim}
## [1] 0.2133933
\end{verbatim}

\begin{enumerate}
\def\labelenumi{\alph{enumi})}
\setcounter{enumi}{2}
\tightlist
\item
  ¿Cual es el nivel de glucosa esperado de acuerdo cone sta
  distribucion?
\end{enumerate}

\begin{Shaded}
\begin{Highlighting}[]
\NormalTok{(E }\OtherTok{\textless{}{-}}\NormalTok{  alpha }\SpecialCharTok{*}\NormalTok{ beta)}
\end{Highlighting}
\end{Shaded}

\begin{verbatim}
## [1] 76.5
\end{verbatim}

\begin{enumerate}
\def\labelenumi{\arabic{enumi}.}
\setcounter{enumi}{5}
\tightlist
\item
  Sea \((X, Y)\) un vector aleatorio continuo con funcion de densidad
  conjunta \[
  $f(x,y) = \begin{cases}
       cx^2y^3, & 0 < x < 1,0 < y <1,\\
       0, &  \text{otro caso.}
       \end{cases}$
  \]
\end{enumerate}

\begin{Shaded}
\begin{Highlighting}[]
\NormalTok{f\_X\_a }\OtherTok{\textless{}{-}} \ControlFlowTok{function}\NormalTok{(x) \{}
  \ControlFlowTok{if}\NormalTok{ (}\DecValTok{0} \SpecialCharTok{\textless{}}\NormalTok{ x }\SpecialCharTok{\&\&}\NormalTok{ x }\SpecialCharTok{\textless{}} \DecValTok{1}\NormalTok{) \{}
    \FunctionTok{return}\NormalTok{((}\DecValTok{1}\SpecialCharTok{*}\NormalTok{x}\SpecialCharTok{**}\DecValTok{2}\NormalTok{)}\SpecialCharTok{/}\DecValTok{4}\NormalTok{)}
\NormalTok{  \} }\ControlFlowTok{else}\NormalTok{ \{}
    \FunctionTok{return}\NormalTok{(}\DecValTok{0}\NormalTok{)}
\NormalTok{  \}}
\NormalTok{\}}

\NormalTok{f\_Y\_a }\OtherTok{\textless{}{-}} \ControlFlowTok{function}\NormalTok{(y) \{}
  \ControlFlowTok{if}\NormalTok{ (}\DecValTok{0} \SpecialCharTok{\textless{}}\NormalTok{ y }\SpecialCharTok{\&\&}\NormalTok{ y }\SpecialCharTok{\textless{}} \DecValTok{1}\NormalTok{) \{}
    \FunctionTok{return}\NormalTok{((}\DecValTok{1}\SpecialCharTok{*}\NormalTok{y}\SpecialCharTok{**}\DecValTok{3}\NormalTok{)}\SpecialCharTok{/}\DecValTok{3}\NormalTok{)}
\NormalTok{  \} }\ControlFlowTok{else}\NormalTok{ \{}
    \FunctionTok{return}\NormalTok{(}\DecValTok{0}\NormalTok{)}
\NormalTok{  \}}
\NormalTok{\}}

\NormalTok{f\_xy\_a }\OtherTok{\textless{}{-}} \ControlFlowTok{function}\NormalTok{(x, y) \{}
  \ControlFlowTok{if}\NormalTok{ (}\DecValTok{0} \SpecialCharTok{\textless{}}\NormalTok{ x }\SpecialCharTok{\&\&}\NormalTok{ x }\SpecialCharTok{\textless{}} \DecValTok{1} \SpecialCharTok{\&\&} \DecValTok{0} \SpecialCharTok{\textless{}}\NormalTok{ y }\SpecialCharTok{\&\&}\NormalTok{ y }\SpecialCharTok{\textless{}} \DecValTok{1}\NormalTok{) \{}
    \FunctionTok{return}\NormalTok{(}\DecValTok{1} \SpecialCharTok{*}\NormalTok{ y}\SpecialCharTok{**}\DecValTok{2} \SpecialCharTok{*}\NormalTok{ y}\SpecialCharTok{**}\DecValTok{3}\NormalTok{)}
\NormalTok{  \} }\ControlFlowTok{else}\NormalTok{ \{}
    \FunctionTok{return}\NormalTok{(}\DecValTok{0}\NormalTok{)}
\NormalTok{  \}}
\NormalTok{\}}
\NormalTok{x }\OtherTok{=}\NormalTok{ .}\DecValTok{2}
\NormalTok{y }\OtherTok{=}\NormalTok{ .}\DecValTok{4}



\ControlFlowTok{if}\NormalTok{(}\FunctionTok{f\_xy\_a}\NormalTok{(x,y)}\SpecialCharTok{==}\FunctionTok{f\_Y\_a}\NormalTok{(y) }\SpecialCharTok{*} \FunctionTok{f\_X\_a}\NormalTok{(x))\{}
  \FunctionTok{cat}\NormalTok{(}\StringTok{"Son independientes"}\NormalTok{)}
\NormalTok{\}}\ControlFlowTok{else}\NormalTok{\{}
  \FunctionTok{cat}\NormalTok{(}\StringTok{"Son dependientes"}\NormalTok{)}
\NormalTok{\}}
\end{Highlighting}
\end{Shaded}

\begin{verbatim}
## Son dependientes
\end{verbatim}

\end{document}
